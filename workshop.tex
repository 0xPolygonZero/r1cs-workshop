\documentclass{article}

\usepackage{hyperref}
\usepackage{mathtools}

\title{R1CS Programming \\ \large ZK0x04 Workshop Notes}
\author{Daniel Lubarov \and Brendan Farmer}
\date{\today}

\DeclarePairedDelimiter\ceil{\lceil}{\rceil}
\DeclarePairedDelimiter\floor{\lfloor}{\rfloor}

% Hyperlink colors.
\hypersetup{
    colorlinks=true,
    linkcolor=blue,
    citecolor=blue,
    urlcolor=blue,
}

% Tweak autoref names to use capitals.
\renewcommand{\sectionautorefname}{Section}

\begin{document}

\maketitle

{\hypersetup{hidelinks} \tableofcontents}
\newpage


\section{Multiplicative inverse} \label{sec:inverse}

Deterministically computing $1 / x$ in an R1CS circuit would be expensive. Instead, we can have the prover compute $1 / x$ outside of the circuit and supply the result as a witness element, which we will call $x_\mathrm{inv}$. To verify the result, we enforce
\begin{equation}
  (x) (x_\mathrm{inv}) = (1)
\end{equation}


\section{Zero testing}

To assert $x = 0$, we simply enforce
\begin{equation}
  (x) (1) = (0)
\end{equation}
Asserting $x \ne 0$ is similarly easy: we compute $1 / x$ (non-deterministically, as in \autoref{sec:inverse}). The result can be ignored; the mere fact that an inverse exists implies $x \ne 0$.

On the other hand, if we want to \textit{evaluate}
\begin{equation}
  y =
  \begin{cases}
    0 & \text{if $x = 0$} \\
    1 & \text{otherwise}
  \end{cases}
\end{equation}
we can do so by introducing another variable $m$, and enforcing
\begin{align}
  (x) & (m) &= (y) \\
  (1 - y) & (x) &= 0
\end{align}
This method is from \cite{parno2013pinocchio}.


\section{Binary}

To assert $b \in \{ 0, 1 \}$, we enforce
\begin{equation}
  (b) (b - 1) = (0)
\end{equation}


\section{Comparisons}

TODO: Describe basic comparison algorithm

TODO: Describe Ahmed's optimization

A few other optimizations are possible in particular circumstances:
\begin{enumerate}
  \item To assert (not evaluate) $x < y$, we can split $x$ non-canonically and split $y$ canonically. The prover is forced to use $x$'s canonical representation anyway, otherwise $x_\mathrm{bin} \ge |F| > y_\mathrm{bin}$, making the assertion unsatisfiable.
  \item To assert $x < c$ for some constant $c \ll |F|$, we can split $x$ into just $\ceil{\log_2(c)}$ bits.
\end{enumerate}


\section{Permutations}

Say we want to verify that two sequences, $(x_1, \dots, x_n)$ and $(y_1, \dots, y_n)$, are permutations of one another.


\section{Sorting}

TODO: Discuss sorting networks

TODO: Discuss permutation networks + comparisons to verify order


\section{Random access}

TODO: Discuss naive random access via index comparisons

TODO: Discuss binary tree method


\section{Embedded curve operations}

TODO: Discuss basic embedded curve operations


\bibliography{bibliography}{}
\bibliographystyle{ieeetr}

\end{document}
